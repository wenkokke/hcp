\documentclass[aspectratio=169,xcolor={dvipsnames,table}]{beamer}


%% Terms, types, and constraints
\definecolor{tmcolor}{HTML}{C02F1D}
\definecolor{tycolor}{HTML}{1496BB}
\definecolor{cscolor}{HTML}{A3B86C}
\providecommand{\tm}[1]{\textcolor{Red}{\ensuremath{\normalfont#1}}}
\providecommand{\ty}[1]{\textcolor{NavyBlue}{\ensuremath{\normalfont#1}}}
\providecommand{\cs}[1]{\textcolor{Emerald}{\ensuremath{\normalfont#1}}}

%% Proof trees
\usepackage{bussproofs}
\EnableBpAbbreviations

\providecommand{\seq}[2][]{\ensuremath{\tm{#1}\;\vdash\;\ty{#2}}}
\providecommand{\cseq}[3][]{\ensuremath{\seq[{#1}]{#2}\mathbin{\triangleright}\cs{#3}}}
\providecommand{\coh}[2][]{\ensuremath{\tm{#1}\mathbin{\vDash}\ty{#2}}}
\providecommand{\tmty}[2]{\ensuremath{\tm{#1}\colon\!\ty{#2}}}
\providecommand{\NOM}[1]{\RightLabel{\textsc{#1}}}
\providecommand{\SYM}[1]{\RightLabel{\ensuremath{#1}}}
\providecommand{\pvar}[1]{\AXC{#1}\noLine\UIC{$\vphantom{\Gamma}\smash[t]{\vdots}$}\noLine}
\newenvironment{prooftree*}{\leavevmode\hbox\bgroup}{\DisplayProof\egroup}
\newenvironment{scprooftree}[1][1]%
  {\gdef\scalefactor{#1}\begin{center}\proofSkipAmount \leavevmode}%
  {\scalebox{\scalefactor}{\DisplayProof}\proofSkipAmount \end{center} }
\newenvironment{scprooftree*}[1][1]%
  {\gdef\scalefactor{#1}\leavevmode\hbox\bgroup}
  {\DisplayProof\egroup} 
\providecommand{\moveMixDown}{\leadsto}

%% Functions
\providecommand{\co}[1]{\tm{\overline{#1}}}
\providecommand{\pr}[1]{\ensuremath{\cs{\text{pr}(\ty{#1})}}}
\providecommand{\un}[2]{\ensuremath{\cs{\ty{#1} \bowtie \ty{#2}}}}
\providecommand{\fg}[1]{\ensuremath{\ty{\lfloor #1 \rfloor}}}
\providecommand{\lt}[2]{\ensuremath{\cs{\text{pr}^{\cs{#1}}(\ty{#2})}}}
\providecommand{\mtf}[1]{\ensuremath{\llbracket #1 \rrbracket}}
\providecommand{\ftm}[1]{\ensuremath{\llparenthesis #1 \rrparenthesis}}
\providecommand{\hole}{\ensuremath{\Box}}
\providecommand{\fv}[1]{\ensuremath{\text{fv}(#1)}}
\providecommand{\freeIn}[2]{\tm{#1}\in\tm{#2}}
\providecommand{\notFreeIn}[2]{\tm{#1}\not\in\tm{#2}}

%% Term reduction
\providecommand{\reducesto}[3][]{\ensuremath{\tm{#2}\overset{#1}{\Longrightarrow}\tm{#3}}}
\providecommand{\nreducesto}[2]{\ensuremath{\tm{#1}\centernot\Longrightarrow\tm{#2}}}

%% Type systems
%%% Depends on: xspace, amsmath, preamble-typing

\providecommand{\ppar}{\ensuremath{\mid}}

%%% Terms
\providecommand{\piCalc}[0]{\textpi-calculus\xspace}
\providecommand{\piSend}[3]{\ensuremath{#1\langle #2 \rangle.#3}}
\providecommand{\piBoundSend}[3]{\ensuremath{#1[ #2 ].#3}}
\providecommand{\piRecv}[3]{\ensuremath{#1( #2 ).#3}}
\providecommand{\piPar}[2]{\ensuremath{#1 \ppar #2}}
\providecommand{\piNew}[2]{\ensuremath{(\nu #1)#2}}
\providecommand{\piRepl}[1]{\ensuremath{!#1}}
\providecommand{\piHalt}[0]{\ensuremath{0}}
\providecommand{\piSub}[3]{\ensuremath{#3\{#1/#2\}}}

%%% Local Variables:
%%% TeX-master: "main"
%%% End:

%% Depends on: amsmath, stmaryrd, xspace, bussproofs, preamble-typing, preamble-pi

\providecommand{\piDILL}{\textpi DILL\xspace}
\providecommand{\cp}{CP\xspace}
\providecommand{\rcp}{RCP\xspace}

%% Legacy
\providecommand{\cpCutOld}[3]{\ensuremath{\nu #1.(\piPar{#2}{#3})}}

%% Terms
\providecommand{\cpLink}[2]{\ensuremath{#1{\leftrightarrow}#2}}
\providecommand{\cpCut}[3]{\ensuremath{\piNew{#1}({\piPar{#2}{#3}})}}
\providecommand{\cpSend}[4]{\ensuremath{#1[#2].(\piPar{#3}{#4})}}
\providecommand{\cpRecv}[3]{\ensuremath{#1(#2).#3}}
\providecommand{\cpWait}[2]{\ensuremath{#1().#2}}
\providecommand{\cpHalt}[1]{\ensuremath{#1[].0}}
\providecommand{\cpInl}[2]{\ensuremath{#1\triangleleft\texttt{inl}.#2}}
\providecommand{\cpInr}[2]{\ensuremath{#1\triangleleft\texttt{inr}.#2}}
\providecommand{\cpCase}[3]{\ensuremath{#1\triangleright\{\texttt{inl}:#2;\texttt{inr}:#3\}}}
\providecommand{\cpAbsurd}[1]{\ensuremath{#1\triangleright\{\}}}
\providecommand{\cpSub}[3]{\ensuremath{\piSub{#1}{#2}{#3}}}
\providecommand{\cpPlug}[2]{\ensuremath{{#1}[{#2}]}}

%% Types
\providecommand{\parr}{\mathbin{\bindnasrepma}}
\providecommand{\with}{\mathbin{\binampersand}}
\providecommand{\plus}{\ensuremath{\oplus}}
\providecommand{\tens}{\ensuremath{\otimes}}
\providecommand{\one}{\ensuremath{\mathbf{1}}}
\providecommand{\nil}{\ensuremath{\mathbf{0}}}
\providecommand{\limp}{\ensuremath{\multimap}}
\providecommand{\emptycontext}{\ensuremath{\,\cdot\,}}
\providecommand{\bigtens}{\ensuremath{\scalerel*{\tens}{\sum}}}
\providecommand{\bigparr}{\ensuremath{\scalerel*{\parr}{\sum}}}


%% Reduction rule names
\newcommand{\cpEquivLinkComm}{\ensuremath{(\cpLink{}{}\text{-sym})}\xspace}
\newcommand{\cpEquivCutComm}{\ensuremath{(\nu\text{-comm})}\xspace}
\newcommand{\cpEquivCutAssNoParen}[1]{\ensuremath{\nu\text{-assoc}\xspace}}
\newcommand{\cpEquivCutAss}[1]{\ensuremath{(\cpEquivCutAssNoParen{#1})}\xspace}
\newcommand{\cpRedAxCut}[1]{\ensuremath{(\cpLink{}{})}\xspace}
\newcommand{\cpRedBetaTensParr}{\ensuremath{(\beta{\tens}{\parr})}\xspace}
\newcommand{\cpRedBetaOneBot}{\ensuremath{(\beta{\one}{\bot})}\xspace}
\newcommand{\cpRedBetaPlusWith}[1]{\ensuremath{(\beta{\plus}{\with}_{#1})}\xspace}
\newcommand{\cpRedKappaTens}[1]{\ensuremath{(\kappa{\tens}_{#1})}\xspace}
\newcommand{\cpRedKappaParr}{\ensuremath{(\kappa{\parr})}\xspace}
\newcommand{\cpRedKappaBot}{\ensuremath{(\kappa{\bot})}\xspace}
\newcommand{\cpRedKappaPlus}[1]{\ensuremath{(\kappa{\plus}_{#1})}\xspace}
\newcommand{\cpRedKappaWith}{\ensuremath{(\kappa{\with})}\xspace}
\newcommand{\cpRedKappaTop}{\ensuremath{(\kappa{\top})}\xspace}
\newcommand{\cpRedGammaCut}{\ensuremath{(\gamma\nu)}\xspace}
\newcommand{\cpRedGammaEquiv}{\ensuremath{(\gamma{\equiv})}\xspace}

%% Structural rules
\providecommand{\cpInfAx}{%
  \begin{prooftree*}
    \AXC{$\vphantom{\seq[ Q ]{ \Delta, \tmty{y}{A^\bot} }}$}
    \NOM{Ax}
    \UIC{$\seq[ \cpLink{x}{y} ]{ \tmty{x}{A}, \tmty{y}{A^\bot} }$}
  \end{prooftree*}}
\providecommand{\cpInfCut}{%
  \begin{prooftree*}
    \AXC{$\seq[ P ]{ \Gamma, \tmty{x}{A} }$}
    \AXC{$\seq[ Q ]{ \Delta, \tmty{x}{A^\bot} }$}
    \NOM{Cut}
    \BIC{$\seq[ \cpCut{x}{P}{Q} ]{ \Gamma, \Delta }$}
  \end{prooftree*}}
\providecommand{\cpInfMix}{%
  \begin{prooftree*}
    \AXC{$\seq[ P ]{ \Gamma }$}
    \AXC{$\seq[ Q ]{ \Delta }$}
    \NOM{Mix}
    \BIC{$\seq[ \piPar{P}{Q} ]{ \Gamma , \Delta }$}
  \end{prooftree*}}
\providecommand{\cpInfCycle}{%
  \begin{prooftree*}
    \AXC{$\seq[ P ]{ \Gamma, \tmty{x}{A}, \tmty{y}{A^\bot}}$}
  \end{prooftree*}}
\providecommand{\cpInfHalt}{%
  \begin{prooftree*}
    \AXC{$\vphantom{\seq[ Q ]{ \Delta, \tmty{y}{A^\bot} }}$}
    \NOM{Halt}
    \UIC{$\seq[{ \piHalt }]{}$}
  \end{prooftree*}}

%% Logical rules
\providecommand{\cpInfTens}{%
  \begin{prooftree*}
    \AXC{$\seq[ P ]{ \Gamma , \tmty{y}{A} }$}
    \AXC{$\seq[ Q ]{ \Delta , \tmty{x}{B} }$}
    \SYM{(\tens)}
    \BIC{$\seq[ \cpSend{x}{y}{P}{Q} ]{ \Gamma , \Delta , \tmty{x}{A \tens B} }$}
  \end{prooftree*}}
\providecommand{\cpInfParr}{%
  \begin{prooftree*}
    \AXC{$\seq[ P ]{ \Gamma , \tmty{y}{A} , \tmty{x}{B} }$}
    \SYM{(\parr)}
    \UIC{$\seq[ \cpRecv{x}{y}{P} ]{ \Gamma , \tmty{x}{A \parr B} }$}
  \end{prooftree*}}
\providecommand{\cpInfOne}{%
  \begin{prooftree*}
    \AXC{$\vphantom{\seq[ P ]{ \Gamma }}$}
    \SYM{(\one)}
    \UIC{$\seq[ \cpHalt{x} ]{ \tmty{x}{\one} }$}
  \end{prooftree*}}
\providecommand{\cpInfBot}{%
  \begin{prooftree*}
    \AXC{$\seq[ P ]{ \Gamma }$}
    \SYM{(\bot)}
    \UIC{$\seq[ \cpWait{x}{P} ]{ \Gamma , \tmty{x}{\bot} }$}
  \end{prooftree*}}
\providecommand{\cpInfPlus}[1]{%
  \ifdim#1pt=1pt
  \begin{prooftree*}
    \AXC{$\seq[ P ]{ \Gamma , \tmty{x}{A} }$}
    \SYM{(\plus_1)}
    \UIC{$\seq[{ \cpInl{x}{P} }]{ \Gamma , \tmty{x}{A \plus B} }$}
  \end{prooftree*}
  \else%
  \ifdim#1pt=2pt
  \begin{prooftree*}
    \AXC{$\seq[ P ]{ \Gamma , \tmty{x}{B} }$}
    \SYM{(\plus_2)}
    \UIC{$\seq[ \cpInr{x}{P} ]{ \Gamma , \tmty{x}{A \plus B} }$}
  \end{prooftree*}
  \else%
  \fi%
  \fi%
}
\providecommand{\cpInfWith}{%
  \begin{prooftree*}
    \AXC{$\seq[ P ]{ \Gamma , \tmty{x}{A} }$}
    \AXC{$\seq[ Q ]{ \Gamma , \tmty{x}{B} }$}
    \SYM{(\with)}
    \BIC{$\seq[ \cpCase{x}{P}{Q} ]{ \Gamma , \tmty{x}{A \with B} }$}
  \end{prooftree*}}
\providecommand{\cpInfNil}{%
  \text{(no rule for \ty{\nil})}}
\providecommand{\cpInfTop}{%
  \begin{prooftree*}
    \AXC{}
    \SYM{(\top)}
    \UIC{$\seq[ \cpAbsurd{x} ]{ \Gamma, \tmty{x}{\top} }$}
  \end{prooftree*}}

%%% Local Variables:
%%% TeX-master: "main"
%%% End:

%% Depends on: xspace, amsmath, bussproofs, rotating, preamble-typing, preamble-cp

\providecommand{\dhcp}{HCP\xspace}
\providecommand{\hcp}{HCP$^{{-}}$\xspace}
\providecommand{\hsep}{\ensuremath{\mid}}
\providecommand{\emptyhypercontext}{\varnothing}


%% Names of rules
\providecommand{\hcpEquivLinkComm}{\cpEquivLinkComm}
\providecommand{\hcpEquivCutComm}{\cpEquivCutComm}
\providecommand{\hcpEquivCutAss}[1]{\cpEquivCutAss{#1}}
\providecommand{\hcpEquivMixComm}{\ensuremath{({\ppar}\text{-comm})}\xspace}
\providecommand{\hcpEquivMixAssNoParen}[1]{\ensuremath{{\ppar}\text{-assoc}}\xspace}
\providecommand{\hcpEquivMixAss}[1]{\ensuremath{(\hcpEquivMixAssNoParen{#1})}\xspace}
\providecommand{\hcpEquivMixHaltNoParen}[1]{\ensuremath{\text{halt}\xspace}}
\providecommand{\hcpEquivMixHalt}[1]{\ensuremath{(\hcpEquivMixHaltNoParen{#1})}\xspace}
\providecommand{\hcpEquivMixCut}[1]{\ensuremath{(\text{scope-ext})}\xspace}




%% Logical rules

%% Depends on: preamble-typing, preamble-pi, preamble-hcp

\providecommand{\hcp}{HCP\xspace}


%% Names of rules
\providecommand{\hcpEquivMixComm}{%
  \ensuremath{({\ppar}\text{-comm})}\xspace}
\providecommand{\hcpEquivMixAssNoParen}[1]{%
  \ensuremath{{\ppar}\text{-assoc}}\xspace}
\providecommand{\hcpEquivMixAss}[1]{%
  \ensuremath{(\hcpEquivMixAssNoParen{#1})}\xspace}
\providecommand{\hcpEquivScopeExtNoParen}[1]{%
  \ensuremath{\text{scope-ext}\xspace}}
\providecommand{\hcpEquivScopeExt}[1]{%
  \ensuremath{(\hcpEquivScopeExtNoParen{#1})}\xspace}
\providecommand{\hcpEquivMixHaltNoParen}[1]{%
  \hcpEquivMixHaltNoParen{#1}}
\providecommand{\hcpEquivMixHalt}[1]{%
  \hcpEquivMixHalt{#1}}
\providecommand{\hcpEquivNewComm}{%
  \ensuremath{(\nu\text{-comm})}\xspace}
\providecommand{\hcpEquivDelay}{%
  \ensuremath{(\text{delay})}}

\providecommand{\hcpRedAxCut}[1]{%
  \ensuremath{(\cpLink{}{})}\xspace}
\providecommand{\hcpRedBetaTensParr}{%
  \ensuremath{(\beta{\tens}{\parr})}\xspace}
\providecommand{\hcpRedBetaOneBot}{%
  \ensuremath{(\beta{\one}{\bot})}\xspace}
\providecommand{\hcpRedBetaPlusWith}[1]{%
  \ensuremath{(\beta{\plus}{\with}_{#1})}\xspace}
\providecommand{\hcpRedGammaNew}{%
  \ensuremath{(\gamma{\nu})}\xspace}
\providecommand{\hcpRedGammaMix}{%
  \ensuremath{(\gamma{\ppar})}\xspace}
\providecommand{\hcpRedGammaEquiv}{%
  \ensuremath{(\gamma{\equiv})}\xspace}

%% Structural rules
\providecommand{\hcpInfAx}{\cpInfAx}
\providecommand{\hcpInfCut}{%
  \begin{prooftree*}
    \AXC{$\seq[{ P }]{
        \mathcal{G}\hsep
        \Gamma, \tmty{x}{A} \hsep
        \Delta, \tmty{x}{A^\bot}
      }$}
    \AXC{$\tm{x} \not\in \ty{\mathcal{G}}$}
    \NOM{H-Cut}
    \BIC{$\seq[{ \piNew{x}{P} }]{
        \mathcal{G}\hsep
        \ty{\Gamma}, \ty{\Delta}
      }$}
  \end{prooftree*}}
\providecommand{\hcpInfMix}{%
  \begin{prooftree*}
    \AXC{$\seq[ P ]{\mathcal{G} }$}
    \AXC{$\seq[ Q ]{\mathcal{H} }$}
    \NOM{H-Mix}
    \BIC{$\seq[ \piPar{P}{Q} ]{
        \mathcal{G} \hsep \mathcal{H} }$}
  \end{prooftree*}}
\providecommand{\hcpInfHalt}{%
  \begin{prooftree*}
    \AXC{$\vphantom{\seq[ Q ]{ \Delta, \tmty{y}{A^\bot} }}$}
    \NOM{H-Mix$_0$}
    \UIC{$\seq[{ \piHalt }]{ \emptyhypercontext }$}
  \end{prooftree*}}


%% Logical rules
\providecommand{\hcpInfBoundTens}{%
  \begin{prooftree*}
    \AXC{$\seq[{ P }]{
        \ty{\Gamma}, \tmty{y}{A} \hsep
        \ty{\Delta}, \tmty{x}{B}
      }$}
    \SYM{(\tens)}
    \UIC{$\seq[{ \piBoundSend{x}{y}{P} }]{
        \ty{\Gamma}, \ty{\Delta}, \tmty{x}{A \tens B}
      }$}
  \end{prooftree*}}
\providecommand{\hcpInfParr}{%
  \begin{prooftree*}
    \AXC{$\seq[ P ]{%
        \Gamma , \tmty{y}{A} , \tmty{x}{B} }$}
    \SYM{(\parr)}
    \UIC{$\seq[ \cpRecv{x}{y}{P} ]{
        \Gamma , \tmty{x}{A \parr B} }$}
  \end{prooftree*}}
\providecommand{\hcpInfOne}{%
  \begin{prooftree*}
    \AXC{$\seq[{ P }]{ \emptyhypercontext }$}
    \SYM{(\one)}
    \UIC{$\seq[{ \piBoundSend{x}{}{P} }]{
        \tmty{x}{\one}
      }$}
  \end{prooftree*}}
\providecommand{\hcpInfBot}{%
  \begin{prooftree*}
    \AXC{$\seq[ P ]{
        \Gamma }$}
    \SYM{(\bot)}
    \UIC{$\seq[ \cpWait{x}{P} ]{
        \Gamma , \tmty{x}{\bot} }$}
  \end{prooftree*}}
\providecommand{\hcpInfPlus}[1]{%
  \ifdim#1pt=1pt
  \begin{prooftree*}
    \AXC{$\seq[ P ]{
        \Gamma , \tmty{x}{A} }$}
    \SYM{(\plus_1)}
    \UIC{$\seq[{ \cpInl{x}{P} }]{
        \Gamma , \tmty{x}{A \plus B} }$}
  \end{prooftree*}
  \else%
  \ifdim#1pt=2pt
  \begin{prooftree*}
    \AXC{$\seq[ P ]{
        \Gamma , \tmty{x}{B} }$}
    \SYM{(\plus_2)}
    \UIC{$\seq[ \cpInr{x}{P} ]{
        \Gamma , \tmty{x}{A \plus B} }$}
  \end{prooftree*}
  \else%
  \fi%
  \fi%
}
\providecommand{\hcpInfWith}{%
  \begin{prooftree*}
    \AXC{$\seq[ P ]{
        \Gamma , \tmty{x}{A} }$}
    \AXC{$\seq[ Q ]{
        \Gamma, \tmty{x}{B} }$}
    \SYM{(\with)}
    \BIC{$\seq[ \cpCase{x}{P}{Q} ]{
        \Gamma , \tmty{x}{A \with B} }$}
  \end{prooftree*}}
\providecommand{\hcpInfNil}{\cpInfNil}
\providecommand{\hcpInfTop}{%
  \begin{prooftree*}
    \AXC{}
    \SYM{(\top)}
    \UIC{$\seq[ \cpAbsurd{x} ]{ \Gamma, \tmty{x}{\top} }$}
  \end{prooftree*}}

%%% Local Variables:
%%% TeX-master: "main"
%%% End:


%%% Local Variables:
%%% TeX-master: "main"
%%% End:

\usepackage{graphicx}
\usepackage{tikz}

%%% Emoji:
\DeclareGraphicsRule{.ai}{pdf}{.ai}{}
\newcommand{\emoji}[2][1em]{\ensuremath{\vcenter{%
\hbox{\includegraphics[width=#1]{#2}}}}\xspace}
\newcommand{\twemoji}[2][1em]{\emoji[#1]{twemoji/2/assets/#2.ai}}
\newcommand{\mary}[1][1em]{\twemoji[#1]{1f469-1f3fb}}
\newcommand{\john}[1][1em]{\twemoji[#1]{1f466-1f3fe}}
\newcommand{\sliceofcake}[1][1em]{\twemoji[#1]{1f370}}
\newcommand{\doughnut}[1][1em]{\twemoji[#1]{1f369}}
\newcommand{\cake}[1][1em]{\twemoji[#1]{1f382}}
\newcommand{\oldcake}[1][1em]{\emoji[#1]{oldcake.ai}}
\newcommand{\nope}[1][1em]{\twemoji[#1]{1f64c}}
\newcommand{\money}[1][1em]{\twemoji[#1]{1f4b0}}
\newcommand{\bill}[1][1em]{\twemoji[#1]{1f4b7}}
\newcommand{\bank}[1][1em]{\twemoji[#1]{1f3e6}}
\newcommand{\store}[1][1em]{\twemoji[#1]{1f3ea}}
\newcommand{\alice}[1][1em]{\twemoji[#1]{1f467-1f3fd}}
\newcommand{\bob}[1][1em]{\twemoji[#1]{1f474-1f3fc}}
\newcommand{\bankjohn}{\ensuremath{\bank_{\john[0.7em]}}\xspace}
\newcommand{\bankmary}{\ensuremath{\bank_{\mary[0.7em]}}\xspace}

%%% Local Variables:
%%% TeX-master: "main"
%%% End:

\usetheme{metropolis}
\title{Taking Apart Classical Processes}
\date{December 18th, 2017}
\author{Wen Kokke}
\institute{University of Edinburgh}
\newcommand*{\comment}[1]{\textcolor{mLightGreen}{#1}}
\begin{document}
\maketitle

\begin{frame}{Dramatis person\ae}
 \centering\Huge
 \begin{tabular}{lll}
   \mary  && Mary           \\
   \john  && John           \\
   \only<1,3>{\cake}\only<2>{\oldcake}
          && Cake           \\
   \money && Money
 \end{tabular}
\end{frame}

{
  \setbeamertemplate{frame footer}{%
    \textcolor{mDarkTeal}{(Milner, Parrow, and Walker, 1992.)}}
  \begin{frame}{Cake so Good It Should Be Illegal}
    \centering\Huge
    \vfill
    \textit{``Gimme the cake, and you'll have your money!''}
    \textit{``No! Money first!''}
    \vfill
    \[
      \piNew{x}{(\piPar{
          \; \piRecv{x}{z}{\piSend{x}{\money}{\mary}} \;
        }}{
        \; \piRecv{x}{y}{\piSend{x}{\cake}{\john}} \;
        )}
    \]
    \vfill
  \end{frame}
}

{
  \setbeamertemplate{frame footer}{%
    \textcolor{mDarkTeal}{(Caires and Pfenning, 2010\hsep Wadler, 2012.)}}
  \begin{frame}{Classical Processes -- Types, Contexts, and Typing Rules}
    \centering\Large
    \[\!
      \arraycolsep=0.16667em
      \begin{array}{lll}
        \text{Type}&\ty{A},\ty{B}
        & := \ty{A \tens B} \mid \ty{A \parr B}
          \mid \dots \\
        \text{Ctxt}&\ty{\Gamma},\ty{\Delta}
        & := \tmty{x_1}{A_1},\dots,\tmty{x_n}{A_n}
      \end{array}
    \] 
    \only<1>{%
      \begin{center}
        \cpInfAx
        \cpInfCut
      \end{center}
    }
    \only<2>{%
      \begin{center}
        \cpInfTens
        \cpInfParr
      \end{center}
      \begin{center}
        \cpInfOne
        \cpInfBot
      \end{center}
    }
    \vfill
  \end{frame}
}

\begin{frame}{Classical Processes -- Best Facilitator of Illegal Cake Resale}
  \centering
  \vfill
  \Huge
  \textit{``Fine! I'll go first!''}
  \vfill
  \Large
  \[\!
    \begin{aligned}
      \piNew{x}{}
      & ( & \cpSend{x}{u}{\cpLink{u}{\money}}{\cpRecv{x}{z}{\mary}}
          & \piPar{}{}
          & \cpRecv{x}{y}{\cpSend{x}{v}{\cpLink{v}{\cake}}{\john}}
          )
    \end{aligned}
  \]
  \vfill
\end{frame}

\begin{frame}{Classical Processes -- Terms}
  \centering\Large
  \vfill
  \[\!
    \begin{aligned}
      \tm{P}, \tm{Q}, \tm{R}
           :=& \; \tm{\cpCut{x}{P}{Q}}
      &      &\comment{Communication}
      \\ \mid& \; \tm{\cpSend{x}{y}{P}{Q}}
      &      &\comment{Independence, ``send''}
      \\ \mid& \; \tm{\cpRecv{x}{y}{P}}
      &      &\comment{Interdependence, ``receive''}
      \\ \mid& \; \tm{\cpHalt{x}}
      &      &\comment{Halt}
      \\ \mid& \; \tm{\cpWait{x}{P}}
      &      &\comment{Wait}
      \\ \mid& \; \dots
    \end{aligned}
  \]  
\end{frame}

\begin{frame}[label=hccp1]{\hccp\ -- Types, Contexts, and Typing Rules}
  \centering\Large
  \vfill
  \[\!
    \arraycolsep=0.16667em
    \begin{array}{lll}
      \text{Type} & \ty{A},\ty{B}
      & := \ty{A \tens B} \mid \ty{A \parr B}
        \mid \ty{\one} \mid \ty{\bot}
        \mid \dots
      \\
      \text{Ctxt} & \ty{\Gamma},\ty{\Delta}
      & := \tmty{x_1}{A_1},\dots,\tmty{x_n}{A_n}
      \\
      \text{I'm gonna talk about hypersequent CP, about constrained cyclic Meta} & \ty{X},\ty{Y}
      & := \ty{\Gamma_1\hsep\dots\hsep\Gamma_n}
    \end{array}
  \]
  \begin{center}
    \hccpInfAx
    \hccpInfHalt
  \end{center}
  \begin{center}
    \hccpInfCycle
    \hccpInfMix
  \end{center}
\end{frame}

\begin{frame}{\hccp\ -- An Intuition for Semicolons}
  \centering\Large
  If $\tm{P}$ is typed by $\seq[P]{\ty{\Gamma}}$,\\
  it will reduce to a single process.
  \vfill
  If $\tm{P}$ is typed by $\seq[P]{\ty{\Gamma_1}\hsep\dots\hsep\ty{\Gamma_n}}$,\\
  it will reduce to a series of $n$ parallel processes.
\end{frame}

\begin{frame}{\hccp\ -- Cut is Derivable}
  \centering\Large
  \vfill
  \begin{prooftree}
    \AXC{$\seq[P]{\tmty{x}{A}, \Gamma\only<2>{\hsep X}}$}
    \AXC{$\seq[Q]{\tmty{\bar{x}}{A^\bot}, \Delta\only<2>{\hsep Y}}$}
    \NOM{Mix}
    \BIC{$\seq[(\piPar{P}{Q})]{%
        \tmty{x}{A}, \Gamma\hsep \tmty{\bar{x}}{A^\bot},
        \Delta\only<2>{\hsep X\hsep Y}}$}
    \NOM{Cycle}
    \UIC{$\seq[\piNew{x\bar{x}}{(\piPar{P}{Q})}]{%
        \Gamma, \Delta\only<2>{\hsep X\hsep Y}}$}
  \end{prooftree}
\end{frame}

\begin{frame}{\hccp\ -- ``Multicut'' is Derivable}
  \centering\Large
  \vfill
  ``Multicut''
  \begin{prooftree}
    \AXC{$\seq[P]{
        \tmty{x_1}{A_1}, \Gamma_1\hsep
        \dots\hsep
        \tmty{x_n}{A_n}, \Gamma_n\hsep X}$}
    \AXC{$\seq[Q]{
        \tmty{\bar{x}_1}{A^\bot_1}, \Delta_1\hsep
        \dots\hsep
        \tmty{\bar{x}_n}{A^\bot_n}, \Delta_n\hsep Y}$}
    \BIC{$\seq[\piNew{x_1\bar{x}_1 \dots x_n\bar{x}_n}{(\piPar{P}{Q})}]{
        \Gamma_1, \Delta_1\hsep
        \dots\hsep 
        \Gamma_n, \Delta_n\hsep X\hsep Y}$}
  \end{prooftree}
\end{frame}

\begin{frame}[label=hccp2]{\hccp\ -- Types, Contexts, and Typing Rules (cont'd)}
  \centering\Large
  \vfill
  \[\!
    \arraycolsep=0.16667em
    \begin{array}{lll}
      \text{Type} & \ty{A},\ty{B}
      & := \ty{A \tens B} \mid \ty{A \parr B}
        \mid \ty{\one} \mid \ty{\bot}
        \mid \dots
      \\
      \text{Ctxt} & \ty{\Gamma},\ty{\Delta}
      & := \tmty{x_1}{A_1},\dots,\tmty{x_n}{A_n}
      \\
      \text{Meta} & \ty{X},\ty{Y}
      & := \ty{\Gamma_1\hsep\dots\hsep\Gamma_n}
    \end{array}
  \]
  \begin{center}
    \hccpInfBoundTens
    \hccpInfParr
  \end{center}
  \begin{center}
    \hccpInfOne
    \hccpInfBot
  \end{center} 
  \vfill
\end{frame}

\begin{frame}{\hccp\ -- 41st Best Facilitator of Illegal Cake Resale}
  \centering\Huge
  \vfill
  \textit{``Fine! I'll go first!''}
  \vfill
  \[\!
    \begin{aligned}
      \piNew{x\bar{x}}{(\piPar
      {& \; \cpSend{x}{u}{\cpLink{u}{\money}}{\cpRecv{x}{z}{\mary}} \\}
      {& \; \cpRecv{\bar{x}}{y}{\cpSend{\bar{x}}{v}{\cpLink{v}{\cake}}{\john}} \;})}
    \end{aligned}
  \]
  \vfill
\end{frame}

\begin{frame}{\hccp\ -- Terms}
  \centering\Large
  \vfill
  \[\!
    \begin{aligned}
      \tm{P}, \tm{Q}, \tm{R}
           :=& \; \tm{\piNew{x\bar{x}}{P}}
      &      &\comment{Channel Creation}
      \\ \mid& \; \tm{(\piPar{P}{Q})}
      &      &\comment{Parallel Composition}
      \\ \mid& \; \tm{\piHalt}
      &      &\comment{Halted Process}
      \\ \mid& \; \tm{\piBoundSend{x}{y}{P}}
      &      &\comment{Independence, ``send''}
      \\ \mid& \; \tm{\piRecv{x}{y}{P}}
      &      &\comment{Interdependence, ``receive''}
      \\ \mid& \; \dots
    \end{aligned}
  \]  
\end{frame}

{
  \setbeamertemplate{frame footer}{%
    \textcolor{mDarkTeal}{(Boreale, 1996\hsep Lindley and Morris, 2015.)}}
  \begin{frame}{\hccp\ -- Setting Send Free}
    \centering\Large
    \vfill
    \[
      \tm{\piBoundSend{x}{y}{P}} :=
      \tm{\piNew{y\bar{y}}{( \piSend{x}{y}{P} )}}
    \]
    \vfill
    \begin{center}
      \hccpInfTens
    \end{center}
    \vfill
    \begin{prooftree}
      \AXC{$\seq[P]{\tmty{y}{A}, \Gamma\hsep \tmty{x}{B}, \Delta\hsep X}$}
      \SYM{\tens}
      \UIC{$\seq[{\piSend{x}{y}{P}}]{
          \tmty{y}{A}, \Gamma\hsep
          \tmty{x}{A \tens B}, \tmty{\bar{y}}{A^\bot}, \Delta\hsep X}$}
      \NOM{Cycle}
      \UIC{$\seq[{\piNew{y\bar{y}}{( \piSend{x}{y}{P} )}}]{
          \tmty{x}{A \tens B}, \Gamma, \Delta\hsep X}$}
    \end{prooftree}
    \vfill
  \end{frame}
}

\begin{frame}{Cake so Good It Should Be Illegal}
  \centering\Huge
  \vfill
  \textit{``Fine! I'll go first!''}
  \vfill
  \[
    \piNew{x}{(\piPar{
      \; \piSend{x}{\money}{\piRecv{x}{z}{\mary}} \;
    }}{
      \; \piRecv{x}{y}{\piSend{x}{\cake}{\john}} \;
    )}
  \]
  \vfill
\end{frame}

{
  \setbeamertemplate{frame footer}{%
    \textcolor{mDarkTeal}{(*Modulo \emph{a lot of things}, like the case
      construct, etc.)}}
  \begin{frame}{Conclusion?}
    \centering\Huge
    We have taken CP apart, and its term constructs now match that of the
    \textpi-calculus, more or less*!
  \end{frame}
}

{
  \begin{frame}{\hccp\ -- An Interesting Theorem\dots}
    \centering\Large
    If $\seq[P]{\Gamma_1\hsep \dots\hsep \Gamma_{n+1}}$,\\
    then there exist $\tm{x_1}\dots\tm{x_n}$ and $\tm{\pi_1}\dots\tm{\pi_{n+1}}$,
    such that
    \[
      \seq[\piNew{x_1\bar{x_1}}\dots\piNew{x_n\bar{x_n}}{%
        (\pi_1.0 \mid \dots \mid \pi_{n+1}.0)}]{\Gamma_1\hsep \dots\hsep \Gamma_{n+1}}.
    \]
  \end{frame}
}

\begin{frame}{\hccp\ -- \dots and its Awkward Cousin}
  \centering\Large
  \only<1>{%
    If $\seq[P]{\Gamma_1\hsep \dots\hsep \Gamma_{n+1}}$,\\
    then there exist $\tm{x_1}\dots\tm{x_n}$ and $\tm{\pi_1}\dots\tm{\pi_{n+1}}$,
    such that
    \[
      \seq[\piNew{x_1\bar{x_1}}\dots\piNew{x_n\bar{x_n}}{%
        (\{\pi_1,\dots,\pi_{n+1}\}.(0^{n+1}) )}]{\Gamma_1\hsep \dots\hsep \Gamma_{n+1}},
    \]
    where $\tm{(0^n)}$ represents $n$ halted processes in parallel.
  }
  \only<2>{%
    So, for instance
    \[
      \tm{\piNew{x\bar{x}}{(%
          \piPar{\piRecv{x}{}{\piHalt}}{\piSend{\bar{x}}{}{\piHalt}} )}}
    \]
    corresponds to
    \[
      \tm{\piNew{x\bar{x}}{(%
          \piRecv{x}{}{\piSend{\bar{x}}{}{(\piPar{\piHalt}{\piHalt})}})}}
    \]
    which looks deadlocked to me.
  }
\end{frame} 

{
  \begin{frame}{Where to go from here?}
    \centering\Large
    We can submit to a nice, sunny conference upstate\dots
  \end{frame}
}

{
  \setbeamertemplate{frame footer}{%
    \textcolor{mDarkTeal}{(Dardha and Gay, 2017)}}
  \begin{frame}{Where to go from here?}
    \centering\Large
    \only<1>{
      We can restrict the type system further, to restrict access to named
      channels if their co-names are still in scope\dots\\
    }
    \only<2>{
      This gets rid of stuff like
      \[
        \tm{\piNew{x\bar{x}}{(\piNew{y\bar{y}}{(%
              \piRecv{x}{}{\piSend{\bar{x}}{}{%
                  \piRecv{y}{}{\piSend{\bar{y}}{}{(%
                      \piPar{\piHalt}{\piHalt})}})}}}}}.
      \]
    }
  \end{frame}
}

{
  \setbeamertemplate{frame footer}{%
    \textcolor{mDarkTeal}{(Bellin and Scott, 1994)}}
  \begin{frame}{Where to go from here?}
    \centering\Large
    \only<1>{
      We can equate all these processes,
      \[
        \tm{\pi.(P \mid Q)} \equiv \tm{(\pi.P \mid Q)}
        \quad \text{if} \quad \text{fv}(\tm{\pi}) \cap \text{fv}(\tm{Q})
        = \varnothing
      \]
      which allows all deadlocked processes to proceed.
    }
    \only<2>{
      Basically, this is like saying
      \[
        \tm{\piNew{x\bar{x}}{(%
            \piRecv{x}{}{\piSend{\bar{x}}{}{(\piPar{\piHalt}{\piHalt})}})}}
      \]
      is equivalent to
      \[
        \tm{\piNew{x\bar{x}}{(%
            \piPar{\piRecv{x}{}{\piHalt}}{\piSend{\bar{x}}{}{\piHalt}} )}},
      \]
      so it can reduce.
    }
  \end{frame}
}

\begin{frame}{\hccp\ -- Reduction Rules}
  \centering\Large
  \[
    \begin{array}{llll}
      & \tm{\piNew{x\bar{x}}(\piPar{\cpLink{w}{x}}{P})}
      & \Longrightarrow \;
      & \tm{\cpSub{w}{\bar{x}}{P}} 
      \\
      & \tm{\piNew{x\bar{x}}(%
        \piPar{\piRecv{x}{z}{R}}{\piSend{\bar{x}}{y}{P}})}
      & \Longrightarrow \;
      & \tm{\piNew{x\bar{x}}(\piPar{P}{R})}
      \\
      & \tm{\piNew{x\bar{x}}(%
        \piPar{\piRecv{x}{}{R}}{\piSend{\bar{x}}{}{P}})}
      & \Longrightarrow \;
      & \tm{\piNew{x\bar{x}}(\piPar{P}{R})}
    \end{array}
  \]
  \quad
  \begin{prooftree*}
    \AXC{\reducesto{P}{P^\prime}}
    \UIC{\reducesto{\piNew{x\bar{x}}{P}}{\piNew{x\bar{x}}{P^\prime}}}
  \end{prooftree*}
  \begin{prooftree*}
    \AXC{\reducesto{P}{P^\prime}}
    \UIC{\reducesto{( \piPar{P}{Q} )}{( \piPar{P^\prime}{Q} )}}
  \end{prooftree*}
  \begin{prooftree}
    \AXC{$\tm{P}\equiv\tm{Q}$}
    \AXC{\reducesto{Q}{Q^\prime}}
    \AXC{$\tm{Q^\prime}\equiv\tm{P^\prime}$}
    \TIC{\reducesto{P}{P^\prime}}
  \end{prooftree}
\end{frame}

\begin{frame}{\hccp\ -- Structural Congruence}
  \centering\Large
  Where $\equiv$ is reflexive, transitive, congruent, and has:
  \[
    \begin{array}{llll}
        \tm{\cpLink{x}{y}}
      & \equiv \;
      & \tm{\cpLink{y}{x}}
      \\
        \tm{(\piPar{P}{Q})}
      & \equiv \;
      & \tm{(\piPar{Q}{P})}
      \\
        \tm{( \piPar{P}{( \piPar{Q}{R} )} )}
      & \equiv \;
      & \tm{( \piPar{( \piPar{P}{Q} )}{R} )}
      \\
        \tm{\piNew{x\bar{x}}{( \piPar{P}{Q} )}}
      & \equiv \;
      & \tm{( \piPar{\piNew{x\bar{x}}{P}}{Q} )}
      & \text{if} \; \tm{x},\tm{\bar{x}}\not\in\text{fv}(\tm{Q})
      \\
        \tm{\pi.(P \mid Q)}
      & \equiv \;
      & \tm{(\pi.P \mid Q)}
      & \text{if} \; \text{fv}(\tm{\pi}) \cap \text{fv}(\tm{Q}) = \varnothing
    \end{array}
  \]
\end{frame}

\begin{frame}{Questions? Remarks?}
  \centering\Huge
  I have found the calculus I was looking for,
  but maybe not the calculus I wanted\dots
\end{frame}

\againframe{hccp1}

\againframe{hccp2}

\end{document}