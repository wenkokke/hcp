\begin{definition}\label{def:hcp-bigtens}
  We can interpret hyper-environments as types by collapsing them using a series
  of tensors. In the case of the empty hyper-environment
  $\ty{\emptyhypercontext}$, we interpret this as the unit of tensor.
  \begin{gather*}
    \begin{array}{lcl}
      \ty{\bigtens\emptyhypercontext}
      & = & \ty{\one}
      \\
      \ty{\bigtens(\Gamma_1 \hsep \dots \hsep \Gamma_n)}
      & = & \ty{\bigparr\Gamma_1 \tens \dots \tens \bigparr\Gamma_n}
    \end{array}
  \end{gather*}
\end{definition}
\begin{theorem}\label{thm:hcp-bigtens}
  If $\seq{\mathcal{G}}$ in \hcp, then $\seq{\bigtens\mathcal{G}}$ in \hcp.
\end{theorem}
\begin{proof}
  By case analysis on the structure of $\ty{\mathcal{G}}$.
  \begin{itemize}
  \item
    Case $\ty{\mathcal{G}} = \ty{\emptyhypercontext}$.\\
    By inversion on the derivation of $\seq{\emptyhypercontext}$.
    The only typing derivation ends in \textsc{H-Halt}.\\
    We rewrite as follows:
    \[
      \begin{array}{lcl}
        \AXC{}\NOM{H-Halt}\UIC{$\seq{\emptyhypercontext}$}
        \DisplayProof
        & \Rightarrow
        & \AXC{}\SYM{\one}\UIC{$\seq{\one}$}
          \DisplayProof
      \end{array}
    \] 
  \item
    Case $\ty{\mathcal{G}} = \ty{\Gamma_1 \hsep \dots \hsep \Gamma_n}$.\\
    By induction on the derivation of $\seq{\Gamma_1 \hsep \dots \hsep \Gamma_n}$.
    \begin{itemize}
    \item
      We remove any application of \textsc{H-Mix} where either premise is \textsc{H-Halt}.
    \item
      We propagate the applications of \textsc{H-Mix} downwards to the root of the
      proof, using the rewrite rules in \cref{fig:cc-hmix}, and obtain a
      derivation of the following form:
      \begin{prooftree}
        \pvar{$\rho_1$}\UIC{$\seq{\Gamma_1}$}
        \AXC{$\dots$}
        \pvar{$\rho_n$}\UIC{$\seq{\Gamma_n}$}
        \NOM{H-Mix}
        \TIC{$\seq{\Gamma_1 \hsep \dots \hsep \Gamma_n}$}
      \end{prooftree}
    \item
      We apply \cref{lem:hcp-bigparr} to each sub-proof $\rho_i$ to obtain a
      series of single-formula sequents, joined by applications of \textsc{H-Mix}.
    \item
      We replace the applications of \textsc{H-Mix} with applications of $(\tens)$.
    \end{itemize}
  \end{itemize}
\end{proof}
%%% Local Variables:
%%% TeX-master: "main"
%%% End:
